\documentclass[]{article}
\usepackage{lmodern}
\usepackage{amssymb,amsmath}
\usepackage{ifxetex,ifluatex}
\usepackage{fixltx2e} % provides \textsubscript
\ifnum 0\ifxetex 1\fi\ifluatex 1\fi=0 % if pdftex
  \usepackage[T1]{fontenc}
  \usepackage[utf8]{inputenc}
\else % if luatex or xelatex
  \ifxetex
    \usepackage{mathspec}
  \else
    \usepackage{fontspec}
  \fi
  \defaultfontfeatures{Ligatures=TeX,Scale=MatchLowercase}
\fi
% use upquote if available, for straight quotes in verbatim environments
\IfFileExists{upquote.sty}{\usepackage{upquote}}{}
% use microtype if available
\IfFileExists{microtype.sty}{%
\usepackage[]{microtype}
\UseMicrotypeSet[protrusion]{basicmath} % disable protrusion for tt fonts
}{}
\PassOptionsToPackage{hyphens}{url} % url is loaded by hyperref
\usepackage[unicode=true]{hyperref}
\hypersetup{
            pdfborder={0 0 0},
            breaklinks=true}
\urlstyle{same}  % don't use monospace font for urls
\IfFileExists{parskip.sty}{%
\usepackage{parskip}
}{% else
\setlength{\parindent}{0pt}
\setlength{\parskip}{6pt plus 2pt minus 1pt}
}
\setlength{\emergencystretch}{3em}  % prevent overfull lines
\providecommand{\tightlist}{%
  \setlength{\itemsep}{0pt}\setlength{\parskip}{0pt}}
\setcounter{secnumdepth}{0}
% Redefines (sub)paragraphs to behave more like sections
\ifx\paragraph\undefined\else
\let\oldparagraph\paragraph
\renewcommand{\paragraph}[1]{\oldparagraph{#1}\mbox{}}
\fi
\ifx\subparagraph\undefined\else
\let\oldsubparagraph\subparagraph
\renewcommand{\subparagraph}[1]{\oldsubparagraph{#1}\mbox{}}
\fi

% set default figure placement to htbp
\makeatletter
\def\fps@figure{htbp}
\makeatother


\date{}

\begin{document}

\subsection{Postup riešenia}\label{postup-rieux161enia}

Celá detekcia bola prezatiaľ realizovaná iba na malej sade obrázkov,
myslím, že pamäť RAM by nemala byť problémom, budem to spracovávať iba
po jednotlivých sadách, kde jedna sada má \(\pm\) 150MB, v pamäti to
bude do 500MB potom.

Postup je následovný:

\begin{enumerate}
\def\labelenumi{\arabic{enumi}.}
\tightlist
\item
  Detekcia bodov, ktoré by mohli odpovedať výboju, problém je kvantový
  šum a prekrývajúce sa výboje.
\item
  Matematická erózia, odstránim tak kvantový šum čiastočne, ktorý
  predpokladám, že je bodový.
\item
  Matematická dilatácia, spojí vzdialenejšie body od seba. (Mám zarátať
  aj tvar štrukturálneho elemmentu, kaďže predpokladáme, že výboj je
  vyšší a iba málo široký?).
\item
  Výber iba naozaj jasných oblastí, ktoré sú ``semienkami'' výsledných
  segmentov.
\item
  Výber masky (všetky body, ktoré majú byť priradené k oblastiam).
\item
  Watershed s parametrami:

  \begin{itemize}
  \tightlist
  \item
    Obrázok originálny, naprahovaný.
  \item
    Semienka, ktoré sú očíslované v určitom poradí.
  \item
    Maska s bodmi, ktoré budeme segmentovať (segmentácia celého obrázku
    nemá zmysel)
  \end{itemize}
\item
  Výsledkom je obrázok s oblasťami číslovanými 0 - pozadie, hodnoty
  \(left[1,n\right]\) s jednotlivými segmentovanými oblasťami.
\end{enumerate}

\subsection{Možné vylepšenia}\label{moux17enuxe9-vylepux161enia}

Preteraz ma napadajú niektoré z vylepšení:

\begin{itemize}
\tightlist
\item
  Potlačenie ``semienok'', ktoré majú pod určitou veľkosťou. Problém
  môže byť, že následne nebudeme schopní detekovať niektoré menšie
  výboje teoreticky.
\item
  Odstránenie z finálneho obrázku malé detekované objekty, rovnaký
  problém ako v predošlom príklade.
\item
  Ako nastaviť hodnoty prahu?
\item
  Iná metóda ako watershed?
\end{itemize}

\subsection{TO-DO list}\label{to-do-list}

\begin{itemize}
\tightlist
\item
  {[} {]} Otestovať funkčnosť na výbojoch v druhej polperióde.
\item
  {[}x{]} Detekcia jednotlivých výbojov aspoň na jednom obrázku.
\item
  {[} {]} Načítanie obrázkov v rade.
\end{itemize}

\end{document}
